\documentclass{article}
\usepackage[utf8]{inputenc}
%\setlength{\itemsep}{0pt}
%\setlength{\parsep}{0pt}
%\setlength{\oddsidemargin}{-0.1in}
%\setlength{\evensidemargin}{-0.1in}
%\setlength{\textheight}{9in}
%\setlength{\textwidth}{6in}
\usepackage[margin=1in]{geometry}
\usepackage{graphics}
\usepackage{bm}
\usepackage{verbatim}


\title{663 Final Project: NUTS Algorithm}
\author{Stephanie Brown, Liz Lorenzi, Jody Wortman}
\date{April 6, 2016}

\begin{document}

\maketitle

\section*{Abstract}
Our project aims to implement and explore the No-U-Turn Sampler (NUTS), an algorithm for adaptively setting path lengths in Hamiltonian Monte Carlo. Hamiltonian Monte Carlo (HMC) is an MCMC algorithm that uses first-order gradient information to enable faster convergence by avoiding the random walk behavior and sensitivity to correlated parameters.  Specifically, NUTS uses a recursive algorithm which eliminates the need to manually tune parameters for step size and number of steps.  We implement this algorithm and compare its posterior accuracy and computational efficiency to other MCMC methods, such as Metropolis-Hastings, Gibbs samplers, and Hamiltonian MCMC. We show examples of how the tuning parameters from HMC can result in poor posterior estimations, and how the NUTS recursive tuning results in similar or better estimation without the need for manual tuning. In addition, we implement the Efficient NUTS algorithm as well as the NUTS with Dual Averaging, to explore the speed ups available for this algorithm.


\section{Introduction}
\subsection{MCMC}
\subsection{Hamiltonian Monte Carlo}
\subsection{NUTS Algorithm}
\subsection{NUTS Advantages}



\section{Implementation}
\subsection{Hamiltonian Monte Carlo}
\subsection{Naive No-U-Turn Sampler}


\section{Optimization}
\subsection{Efficient No-U-Turn Sampler}
\subsection{No-U-Turn Sampler with Dual Averaging}


\section{Testing: Comparison to other MCMC methods}
\subsection{Posterior Estimation} 
\subsection{Efficiency}
\subsection{Comparison to Stan}

\section{Discussion}
\subsection{Review of Algorithm}
\end{document}
